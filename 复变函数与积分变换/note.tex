\documentclass{article}

\usepackage{amsmath}
\usepackage{amssymb}
\usepackage[UTF8]{ctex}
\usepackage{geometry}
\usepackage[all]{xy}
\usepackage{hyperref}

\hypersetup{
colorlinks=true,
linkcolor=blue,
anchorcolor=blue,
citecolor=blue}

\title{复变函数复习}
\author{ming}
\date{\today}

\begin{document}

\maketitle

\section{复数}


\subsection{基本展开公式}

\subsubsection{三角式、指数式}

\begin{equation}
    \begin{aligned}
        z &= x + iy \\
          &= r(\cos \theta + i \sin \theta) \\
          &= r e^{i \theta}
    \end{aligned}
\end{equation}

\subsubsection{乘法公式}

\begin{equation}
    \begin{aligned}
        z_1 z_2 &= r_1 r_2 [\cos(r_1+r_2) + i \sin(r_1+r_2)] \\
                &= r_1 r_2 e^{i(r_1+r_2)}
    \end{aligned}
\end{equation}

\subsubsection{次方公式}

\begin{equation}
    z^n = r^n e^{i n \theta} 
\end{equation}

\subsubsection{开方公式}

\begin{equation}
    \sqrt[n]{z} = \sqrt[n]{r} (\cos \frac{\theta + 2k\pi}{n} + i \sin \frac{\theta + 2k\pi}{n}) \quad (k=0,1,2,\cdots,n-1)
\end{equation}

计算步骤:
\begin{enumerate}
    \item 先把根号内的复数转换为三角式
    \item $\cos$ 和 $\sin$ 部分变为 $\frac{\theta + 2k\pi}{n}$ : $k=0,1,2,\cdots,n-1$ ; $n$ 为开$n$次方
\end{enumerate}

\subsection{初等函数}

\subsubsection{指数函数}

\begin{equation}
    e^z = e^x (\cos y + i \sin y) \quad (z=x+iy)
\end{equation}

\begin{equation}
    e^{z_1+z_2} = e^{z_1} e^{z_2} , (e^{z_1})^{z_2} \neq e^{z_1 z_2} , (e^z)^n = e^{nz} \quad (n \in \mathbb{Z})
\end{equation}

\begin{equation}
    Arg(e^z) = y + 2k\pi \quad (k \in \mathbb{Z})
\end{equation}

\begin{equation}
    e^{iz}= \cos z + i \sin z
\end{equation}

\subsubsection{三角函数}

\begin{equation}
    \begin{aligned}
        \sin z &= \frac{1}{2i} (e^{iz} - e^{-iz}) \\
        \cos z &= \frac{1}{2} (e^{iz} + e^{-iz})
    \end{aligned}
\end{equation}

\subsubsection{双曲函数}

\begin{equation}
    \begin{aligned}
        \sinh z &= \frac{1}{2} (e^z - e^{-z}) \\
        \cosh z &= \frac{1}{2} (e^z + e^{-z})
    \end{aligned}
\end{equation}

\subsubsection{对数函数}

\begin{equation}
    \begin{aligned}
        Ln z &= \ln \left\lvert z \right\rvert + i Arg z \quad (z \neq 0,\infty) \\
             &= \ln \left\lvert z \right\rvert + i arg Z + 2k\pi i \quad (k \in \mathbb{Z})
    \end{aligned}
\end{equation}

\subsubsection{幂函数}

\begin{equation}
    z^w = e^{w Ln z}
\end{equation}

\section{解析函数}

\begin{equation}
    f(z) = u(x,y) + iv(x,y)
\end{equation}

\subsection{C-R条件}

$f(z)$ 在区域 $z$ 处可微的充分必要条件是 $u(x,y)$ 和 $v(x,y)$ 在 $z$ 处:

\begin{equation} \label{CR}
    \begin{aligned}
        \frac{\partial u}{\partial x} &= \frac{\partial v}{\partial y} \\
        \frac{\partial u}{\partial y} &= -\frac{\partial v}{\partial x}
    \end{aligned}
\end{equation}

\subsubsection{推论1}

\begin{equation}
    \begin{aligned}
        f'(z) &= \frac{\partial u}{\partial x} + i \frac{\partial v}{\partial x} \\
              &= \frac{\partial v}{\partial y} + i \frac{\partial v}{\partial x} \\
              &= \frac{\partial u}{\partial x} - i \frac{\partial u}{\partial y} \\
              &= \frac{\partial v}{\partial y} - i \frac{\partial u}{\partial y}
    \end{aligned}
\end{equation}

\subsection{解析}

\begin{displaymath}
    \xymatrix{
    [f(z) \text{在} D \text{内解析}] \ar@{<=>} [r] \ar@{=>}[d] & [f(z) \text{在} D \text{内可导}] \ar@{=>}[d] \\
    [f(z) \text{在} z_0 \text{解析}] \ar@{=>}[r] & [f(z) \text{在} z_0 \text{可导}]  
}
\end{displaymath}

\subsection{做题步骤}

判断可导性与解析性

\begin{enumerate}
    \item 写出 $u$ 和 $v$
    \item 求出 $u_x, u_y, v_x, v_y$
    \item 判断是否满足 C-R \eqref{CR} 条件,若满足则可导,否则不可导
    \item 判断是否解析
\end{enumerate}

\section{复积分}

\subsection{类型1:复积分公式}

\begin{equation}
    \oint  f(z) dz = \oint u dx - v dy + i \oint v dx + u dy
\end{equation}

\subsection{类型2:参数方程}

\begin{enumerate}
    \item 线段:$AB, z(t)=A+t(B-A), \quad t \in [0,1]$
    \item 抛物线:$AB,y=x^2,z(t)=t+it^2, \quad t \in [0,1]$
    \item 圆:$z(t)=z_0+R e^{it}, \quad t \in [0,2\pi]$
\end{enumerate}

\begin{equation}
    \oint f(z) dz = \int_{t_1}^{t_2} f(z(t)) z'(t) dt
\end{equation}

\subsubsection{重要结论}

\begin{equation}
    \begin{aligned}
        \oint_{|z-z_0|=r} \frac{dz}{(z-z_0)^n} &= \left\{
            \begin{aligned}
                &2\pi i, \quad n=1 \\
                &0, \quad n \neq 1
            \end{aligned}
        \right. \\
    \end{aligned}
\end{equation}

注意:积分值与 $z_0,r$ 均无关

\subsection{类型3:柯西积分定理}

若 $f(z)$ 在 $C$ 和其内部处处解析,$C$ 为 $D$ 内的任一闭曲线,则:

\begin{equation}
    \oint_C f(z) dz = 0
\end{equation}

\subsection{类型4:牛顿-莱布尼茨公式}

若 $f(z)$ 在 $D$ 内部处处解析,曲线 $C$ 围成单连通区域 $D$,且起点终点在 $D$ 内,则:

\begin{equation}
    \int_{z_1}^{z_2} f(z) dz = F(z_2) - F(z_1), \quad (z_1,z_2 \in D)
\end{equation}

\subsection{类型5:柯西积分公式}

\begin{equation}
    \begin{aligned}
        \oint_C \frac{f(z)}{z-z_0} dz &= 2\pi i f(z_0), \quad (z_0 \in D) \\
        \oint_C \frac{f(z)}{(z-z_0)^{n+1}} dz &= \frac{2\pi i}{n!} f^{(n)}(z_0), \quad (z_0 \in D)
    \end{aligned}
\end{equation}

\section{级数}

\subsection{求收敛半径}

\begin{enumerate}
    \item 根值法:$R = \frac{1}{\varlimsup\limits_{n \to \infty} \sqrt[n]{\left\lvert c_n \right\rvert}}$ ,其中 $\varlimsup$ 代表上极限
    \item 比值法:$R = \lim\limits_{n \to \infty} \left\lvert \frac{c_{n+1}}{c_n} \right\rvert$
\end{enumerate}

\subsection{泰勒级数}

\begin{equation}
    f(z) = \sum_{n=0}^{\infty} \frac{f^{(n)}(z_0)}{n!} (z-z_0)^n
\end{equation}

\subsubsection{常见泰勒级数}

\begin{enumerate}
    \item $e^z = \sum_{n=0}^{\infty} \frac{z^n}{n!}$
    \item $\frac{1}{1-z} = \sum_{n=0}^{\infty} z^n, \quad \left\lvert z \right\rvert < 1$
    \item $\sin z = \sum_{n=0}^{\infty} (-1)^n \frac{ z^{2n+1}}{(2n+1)!}$
    \item $\cos z = \sum_{n=0}^{\infty} (-1)^n \frac{ z^{2n}}{(2n)!}$
    \item $\ln(1+z) = \sum_{n=1}^{\infty} (-1)^{n-1} \frac{ z^n}{n}$
\end{enumerate}

\subsection{洛朗级数}

\begin{equation}
    f(z) = \sum_{n=-\infty}^{\infty} c_n (z-z_0)^n
\end{equation}

\subsection{做题步骤}

\begin{enumerate}
    \item 求奇点,以奇点到展开中心的模的大小为分界
    \item 凑出奇点形式,增补
    \item 分类讨论中“谁大提谁”,相当于把谁塞到分母的分母里
    \item 从 $z_0$ 中心往外扩散,最先接触到奇点的部分先发生变化
\end{enumerate}

\section{留数}

\subsection{奇点}

奇点:无定义点
\begin{enumerate}
    \item 非孤立奇点:$f(z)$ 在 $z_0$ 处无定义,但在 $z_0$ 的某个邻域内有定义。非孤立奇点值为 $\lim\limits_{k \to \infty} z_k$
    \item 孤立奇点:
    \begin{enumerate}
        \item 可去奇点:$f(z)$ 在 $z_0$ 处解析,$\lim\limits_{z \to z_0} f(z)=c_0 \neq \infty$ 
        \item 极点:$f(z)$ 在 $z_0$ 处有限,$\lim\limits_{z \to z_0} (z-z_0)^n f(z)=c_n \neq 0$, $\lim\limits_{z \to z_0} f(z)=\infty$
        \item 本性奇点:$f(z)$ 在 $z_0$ 处无穷,$\lim\limits_{z \to z_0} f(z)$ 不存在
    \end{enumerate}
\end{enumerate}

\subsubsection{孤立奇点判断方法}

\begin{enumerate}
    \item 
    \begin{equation*}
        \text{洛朗展开看负次幂项系数:}
        \left\{
            \begin{aligned}
                0, &\quad z_0 \text{为可去奇点} \\
                m, &\quad z_0 \text{为极点} \\
                \infty, &\quad z_0 \text{为本性奇点}
            \end{aligned}
        \right.
    \end{equation*}
    \item 
    \begin{equation*}
        \text{看} \lim\limits_{z \to z_0} f(z) \text{:}
        \left\{
            \begin{aligned}
                c_0, &\quad z_0 \text{为可去奇点} \\
                \infty, &\quad z_0 \text{为极点} \\
                \text{不存在}, &\quad z_0 \text{为本性奇点}
            \end{aligned}
        \right.
    \end{equation*}
    \item 
    \begin{equation*}
        \text{判断} m \text{阶:}
        \left\{
            \begin{aligned}
                \lim\limits_{z \to z_0} (z-z_0)^m f(z) = c_m \neq 0, \quad (z_0 \neq \infty) \\
                f(z) = \frac{(z-z_0)^m p(z)}{(z-z_0)^n q(z)}, p(z),q(z) \text{在} z_0 \text{解析}, p(z_0),q(z_0) \neq 0, \\
                \text{则此时} z_0 \text{为} 
                \left\{
                    \begin{aligned}
                        m-m, \text{阶极点}, \quad m>n \\
                        \text{可去奇点}, \quad m \leq n
                    \end{aligned}
                \right. \\
            \end{aligned}
        \right.
    \end{equation*}
    \item 
    \begin{equation*}
        z_0\text{是} \frac{1}{f(z)} \text{的} m \text{阶零点} 
    \end{equation*}
\end{enumerate}

\subsubsection{奇点求法}

\begin{enumerate}
    \item 先找出所有奇点
    \item 如存在奇点点列,其极限是非孤立奇点
    \item 孤立奇点类型
\end{enumerate}

\subsection{留数法则}

洛朗级数中 $c_{-1}(z-z_0)^{-1}$ 的系数,称为 $f(z)$ 在孤立奇点 $z_0$ 处的留数,记为 $\text{Res}f(z_0)$

\begin{enumerate}
    \item 设 $z_0$ 是 $f(z)$ 的 1 阶极点,则 $\text{Res}f(z_0) = \lim\limits_{z \to z_0} (z-z_0) f(z)$
    \item 设 $f(z)=\frac{p(z)}{q(z)}$ ,若 $z_0$ 是 $f(z)$ 的 1 阶极点,则 $\text{Res}f(z_0) = \frac{p(z_0)}{q'(z_0)}$
    \item 设 $z_0$ 是 $f(z)$ 的 $m$ 阶极点,则 $\text{Res}f(z_0) = \frac{1}{(m-1)!} \lim\limits_{z \to z_0} \frac{d^{m-1}}{dz^{m-1}} [(z-z_0)^m f(z)]$
    \item $\text{Res}f(\infty) = -\text{Res}[f(\frac{1}{\zeta}) \frac{1}{\zeta^2},0]=-c_{-1}, \quad z=\frac{1}{\zeta}$
\end{enumerate}

\subsection{留数基本定理}

\begin{enumerate}
    \item 若 $f(z)$ 在 $D$ 内除有限个孤立奇点外处处解析,则 $\oint_C f(z) dz = 2\pi i \sum_{k=1}^{n} \text{Res}f(z_k)$
    \item 无穷远点的留数:$\sum_{k=1}^{n} \text{Res}f(z_k) + \text{Res}f(\infty) = 0$
\end{enumerate}

\subsection{做题步骤}

\begin{enumerate}
    \item 求出所有奇点
    \item 判断奇点类型
    \item 求留数
    \item “$ 2 \pi i \times$ 留数和”求积分
\end{enumerate}

\section{拉普拉斯变换}

\subsection{重要变换}

\begin{enumerate}
    \item $L\{f(t)\} = F(s)$
    \item $L\{f'(t)\} = sF(s) - f(0)$
    \item $L\{f''(t)\} = s^2 F(s) - sf(0) - f'(0)$
    \item $L\{e^{at} f(t)\} = F(s-a)$
    \item $L\{t^n f(t)\} = (-1)^n F^{(n)}(s)$
    \item $L\{f(t) * g(t)\} = F(s)G(s)$
\end{enumerate}

\subsection{变换表}

\begin{enumerate}
    \item $L\{1\} = \frac{1}{s}$
    \item $L\{t^n\} = \frac{n!}{s^{n+1}}$
    \item $L\{e^{at}\} = \frac{1}{s-a}$
    \item $L\{\sin \omega t\} = \frac{\omega}{s^2 + \omega^2}$
    \item $L\{\cos \omega t\} = \frac{s}{s^2 + \omega^2}$
    \item $L\{t^n e^{at}\} = \frac{n!}{(s-a)^{n+1}}$
    \item $L\{t^n \sin \omega t\} = \frac{n! \omega}{(s^2 + \omega^2)^{n+1}}$
    \item $L\{t^n \cos \omega t\} = \frac{n! (s^2 - \omega^2)}{(s^2 + \omega^2)^{n+1}}$
    \item $L\{e^{at} \sin \omega t\} = \frac{\omega}{(s-a)^2 + \omega^2}$
    \item $L\{e^{at} \cos \omega t\} = \frac{s-a}{(s-a)^2 + \omega^2}$
    \item $L\{t^n e^{at} \sin \omega t\} = \frac{n! \omega (s-a)}{[(s-a)^2 + \omega^2]^{n+1}}$
    \item $L\{t^n e^{at} \cos \omega t\} = \frac{n! [(s-a)^2 - \omega^2]}{[(s-a)^2 + \omega^2]^{n+1}}$
\end{enumerate}


\end{document}