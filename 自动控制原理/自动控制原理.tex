\documentclass[UTF8]{ctexart}

\usepackage{geometry} % 绘制公式
\usepackage{graphicx}
\usepackage{mathrsfs}%使用花体字母
\usepackage{amsfonts}%使用双线体
\usepackage{mathtools}%使用coloneqq符号
\usepackage{amsmath}
\usepackage{xcolor} % 适用颜色框
\usepackage{tcolorbox}
\usepackage{titling} % 用于调整标题的垂直间距

\geometry{
    a4paper, % 使用A4纸张
    left=20mm, % 左侧页边距为20毫米
    right=20mm, % 右侧页边距为20毫米
    top=25mm, % 顶部页边距为25毫米
    bottom=25mm % 底部页边距为25毫米
}

% 自定义环境“!”代表透明度可调节
\tcbset{
    blue/.style={
        colback=blue!10!white,
        colframe=black,
        boxrule=1pt,
        arc=0mm,
        auto outer arc,
        left=1mm,
        right=1mm,
        top=1mm,
        bottom=1mm
    }
}
\tcbset{
    green/.style={
        colback=green!10!white,
        colframe=black,
        boxrule=1pt,
        arc=0mm,
        auto outer arc,
        left=1mm,
        right=1mm,
        top=1mm,
        bottom=1mm
    }
}

% 调整标题的垂直间距
\setlength{\droptitle}{-60pt}
\pretitle{\begin{center}\LARGE\bfseries\vskip -20pt}
\posttitle{\par\vskip -10pt\rule{\textwidth}{0.4pt}\vskip -10pt\end{center}}
\predate{\begin{center}\large\vskip -10pt}
\postdate{\par\vskip -10pt\rule{\textwidth}{0.4pt}\vskip -20pt\end{center}}

\title{自动控制原理}
\author{章家铭}
\date{\today}

\begin{document}
    \maketitle

    \section{结构图化简}
    
    \begin{enumerate}
        \item 串联等效:$C(s) = G_1(s) G_2(s) R(s)$
        \item 并联等效:$C(s) = [G_1(s) \pm G_2(s)]R(s)$
        \item 反馈等效: $C(s) = \frac{G(s)}{1 \mp G(s)H(s)} R(s)$
        \item 比较点前移 $C(s) = G(s)R_1(s) \pm R_2(s) = [R_1(s) \pm \frac{R_2(s)}{G(s)}]G(s)$
        \item 比较点后移 $C(s) = G(s)[R_1(s) \pm R_2(s)]G(s) = R_1(s)G(s) \pm R_2(s)G(s)$
    \end{enumerate}
    
    \section{梅森公式}

    \begin{equation*}
        p = \frac{1}{\Delta} \sum_{k = 1}^{ n } p_k \Delta_k =G(s)
    \end{equation*}

    \begin{tcolorbox}[green]
    \begin{enumerate}
        \item 源节点:只有输出的节点
        \item 阱节点:只有输入的节点
        \item 混合节点:既有输入也有输出的节点
        \item 前向通道:从源节点到阱节点,且每个节点仅通过一次的通道
        \item 单回路:回路的起点和终点在同一个节点,且每个节点仅通过一次
        \item 不接触回路:两个没有公共点的单回路
    \end{enumerate}
    \end{tcolorbox}

    \begin{enumerate}
        \item 找出所有的前向通道,计算出增益$p_k$(乘积)
        \item 找出所有回路,写出增益 $L_n$  ($L_a$ 为单回路 , $L_b L_c$ 为两两不接触回路,以此类推)
        \item 计算特征式 $\Delta = 1 - \sum L_a + \sum L_b L_c - \sum L_d L_e L_f  ...$
        \item 计算特征余子式 $\Delta_k$ (在$\Delta$中,去除与第$k$条前向通道接触后剩余的回路)
    \end{enumerate}

    \section{稳定性判断——劳斯判据}

    \begin{tcolorbox}[green]
    传递函数:在零初始条件下,输出的拉氏变换与输入的拉式变换的比值
    \begin{equation*}
        G(s) = \frac{ C(s) }{ R(s) }
    \end{equation*}
    分母 = 0 所得的方程为闭环特征方程。\\
    分母 + 分子 = 0 所得的方程为开环特征方程。
    \end{tcolorbox}

    劳斯表列法:

    闭环特征方程为:

    $as^5 + bs^4 + cs^3 + ds^2 + es + f = 0$

    则劳斯表如下:
    \begin{equation*}
        \begin{array}{cccc}
            s^5 & s_{11} = a & s_{12} = c & s_{13} = e\\
            s^4 & s_{21} = b & s_{22} = d & s_{23} = f \\
            s^3 & s_{31} = \frac{s_{12}s_{21}-s_{11}s_{22}}{s_{21}} & s_{32} = \frac{s_{13}s_{21}-s_{11}s_{23}}{s_{21}} \\
            s^2 & s_{41} = \frac{s_{22}s_{31}-s_{21}s_{32}}{s_{31}} & s_{42} = f \\
            s^1 & s_{51} = \frac{s_{32}s_{41}-s_{31}s_{42}}{s_{41}} \\
            s^0 & s_{61} = s_{42}
        \end{array}
    \end{equation*}

    稳定的充要条件为第一列,即$s_{11},s_{21} \cdots s_{61}$均为正,同时,第一列符号改变的次数就是有正实部根的个数。

    其中,当第一列出现0时,则使用无穷小$\varepsilon$来代替。

    若出现一行全为0,则对上一行进行求导,构造辅助方程。

    \section{性能指标}
    \begin{enumerate}
        \item 上升时间$t_t$:指系统输出响应重稳态10\%上升到90\%所用的时间(无震荡)或第一次到稳态的时间(震荡)
        \item 峰值时间$t_p$:指系统输出响应超过稳态值到达第一个峰值所需的时间
        \item 超调量$\sigma \%$:指系统输出响应超出稳态值的最大偏移量占稳态值的百分比 $\sigma \% = \frac{c(t_p)-c(\infty)}{c(\infty)} \times 100 \%$
        \item 调节时间$t_s$:指系统输出响应进入稳态值的 $ \pm 5 \% or \pm 2 \%$ 的误差带时所需的时间。
    \end{enumerate}

    一阶系统:
    \begin{enumerate}
        \item 开环传递函数:$G(s) = \frac{1}{Ts}$
        \item 闭环传递函数:$\Phi(s) = \frac{1}{Ts+1}$
    \end{enumerate}
    
    单位负反馈:
    \begin{enumerate}
        \item 闭环传递函数的分子 = 开环传递函数的分子
        \item 闭环传递函数的分母 = 开环传递函数的分子 + 分母
    \end{enumerate}

    典型无零点二阶系统的开环传递函数 $G(s) = \frac{\omega_n^2}{s(s+2\varepsilon\omega_n)} = \frac{k}{s(Ts+1)} $

    \begin{equation*}
        \begin{aligned}
        k &= \frac{\omega_n}{2\varepsilon}\\
        T &= \frac{1}{2\varepsilon\omega_n}
        \end{aligned}
    \end{equation*}

    稳态误差:
    \begin{equation*}
        e_{ss} = \lim_{t \to \infty} e(t) = \lim_{t \to \infty}[c^\ast  (t) -c(t)]
    \end{equation*}

    \section{根轨迹}

    \begin{tcolorbox}[green]
        开环传递函数:分子为0的点为零点$z_i$(数量为m),分母为0的点为极点$p_i$(数量为n)。
    \end{tcolorbox}
    \begin{enumerate}
        \item 起点和终点:起于开环极点$\times$,终于开环零点$\circ$
        \item 分支数,连续性,对称性:分支数为m,n中大的,根轨迹具有连续性且关于实轴对称。
        \item 分布:若某区域右侧的 n+m 个数为奇数,则该区域是根轨迹
        \item 渐近线:夹角$\phi_a = \frac{(2k+1)\pi}{n-m} , k= 0,1,2,\cdots , n-m-1$ ,交点 $\sigma_a = \frac{\sum p_i \sum z_i}{n-m}$
        \item 分离(会合)点d:$\sum \frac{1}{d-z_i} = \sum \frac{1}{d-p_i}$
        \item 与虚轴交点: 令$s=j\omega$ ,令特征方程实部和虚部分别为0,求解$\omega$
    \end{enumerate}

    \section{串联超前校正}

    \begin{enumerate}
        \item 根据误差函数,求出误差系数$K$
        \item 画出伯德图
        \item 根据图求出$\omega_c$,并求出校正前相位裕度
        \item 根据$\omega_c^\prime$,求出a,T,并以此写出校正装置
        \item 写出校正后传递函数$G = G_0 \cdot G_c$
        \item 验证相位裕度是否满足要求
    \end{enumerate}

\end{document}