\documentclass{article}

\usepackage[UTF8]{ctex}
\usepackage{amsmath}
\usepackage{geometry}

\title{物理实验复习}
\author{ming}
\date{\today}

\begin{document}

\maketitle

\section{绪论}

\subsection{测量值、真值、误差}
测量值:数值 单位

真值:物理量客观存在的值(通常是未知的)

误差:测量值与真值之间的差

\subsection{误差的分类及其特点}

\subsubsection{系统误差}

来源:人员、仪器、方法、环境

\subsubsection{偶然误差}

特点:不可避免、也不可消除,单次测量具有随机性,重复测量服从统计规律

特征:
\begin{itemize}
    \item 单峰性:绝对值小的误差出现的概率大,绝对值大的误差出现的概率小
    \item 对称性:正误差和负误差出现的概率相等
    \item 有界性:大量测量值都集中在真值附近,与真值差异较大的测量值出现的概率几乎为零
    \item 抵偿性:在相同条件下对同一物理量进行多次测量,其偶然误差的算数平均值随测量次数的增加而趋向于零
\end{itemize}


\subsubsection{粗大误差}

特点:不属于正常测量范畴,如果出现在测量结果中,则按一定规定剔除

\subsubsection{精密度、准确度、精确度}

\begin{itemize}
    \item 精密度:数据集中程度,反映随机误差的大小
    \item 准确度:平均值与真值之间的接近程度,反映系统误差的大小
    \item 精确度:精度和准确度的综合反映
\end{itemize}

\subsection{不确定度}

\begin{itemize}
    \item A类不确定度:可以用统计方法估计的分量,通常就是测量平均值的标准偏差$u_A$
    \item B类不确定度:用非统计方法估计的分量$u_B$
    \item 合成不确定度:A类不确定度和B类不确定度的平方和的平方根$u_c$ 
    \item 扩展不确定度:合成不确定度乘以一个因子$k$,$U = k u_c$
\end{itemize}

\subsubsection{直接测量量不确定度评定(公式与计算)}

\begin{itemize}
    \item A类不确定度:$u_A = \sqrt{\frac{\sum(x_i - \bar{x})^2}{n(n-1)}}$
    \item B类不确定度:$u_B = \frac{\Delta x}{\sqrt{3}}$
    \item 合成不确定度:$u_c = \sqrt{u_A^2 + u_B^2}$
    \item 扩展不确定度:$U = 2 u_c$
\end{itemize}

正确结果表达式:

\begin{equation}
    \left\{
    \begin{aligned}
        x &= \bar{x} \pm U \\
        E &= \frac{U}{\bar{x}} \times 100\%
    \end{aligned}
    \right.
\end{equation}

规则:

\begin{itemize}
    \item $\bar{x}$与$U$同单位(数量级)
    \item $\bar{x}$与$U$同指数幂(科学记数法)
    \item $U$只保留一位数字,但计算过程中要多取一位,且$U$作为它值的代入时也保留两位
    \item $\bar{x}$与$U$末尾对齐
    \item $E < 10\%$时,取一位数字,$E \geq 10\%$时,取两位数字
\end{itemize}

\subsection{间接测量量不确定度度评定(公式与计算)}

设间接测量量N与数个直接测量量有函数关系:

\begin{equation}
    N = f(x_1, x_2, \cdots, x_n)
\end{equation}

则:

\begin{equation}
    U_N = \sqrt{\sum_{i=1}^{n} \left(\frac{\partial f}{\partial x_i} U_i\right)^2}
\end{equation}

\begin{equation}
    E = \frac{U_N}{\bar{N}} \times 100\%
\end{equation}

结果表达式为:

\begin{equation}
    \left\{
    \begin{aligned}
        N &= \bar{N} \pm U_N \\
        E &= \frac{U_N}{\bar{N}} \times 100\%
    \end{aligned}
    \right.
\end{equation}

\subsection{有效数字}

\subsubsection{有效数字的运算规则}

\begin{itemize}
    \item 加减法:定末尾
    \item 乘除法:定位数
    \item 有效数字的取舍:4舍6入5凑偶
    \item 乘方、开方:有效数字位数与底数有效数字位数相同
    \item 常数:位数相同
    \item 无理数:比结果多保留一位
\end{itemize}

\end{document}