\documentclass[UTF8]{ctexart}

\usepackage{geometry} % 绘制公式
\usepackage{graphicx}
\usepackage{mathrsfs}%使用花体字母
\usepackage{amsfonts}%使用双线体
\usepackage{mathtools}%使用coloneqq符号
\usepackage{amsmath}
\usepackage{xcolor} % 适用颜色框
\usepackage{tcolorbox}
\usepackage{titling} % 用于调整标题的垂直间距

\geometry{
    a4paper, % 使用A4纸张
    left=20mm, % 左侧页边距为20毫米
    right=20mm, % 右侧页边距为20毫米
    top=25mm, % 顶部页边距为25毫米
    bottom=25mm % 底部页边距为25毫米
}

% 自定义环境“!”代表透明度可调节
\tcbset{
    blue/.style={
        colback=blue!10!white,
        colframe=black,
        boxrule=1pt,
        arc=0mm,
        auto outer arc,
        left=1mm,
        right=1mm,
        top=1mm,
        bottom=1mm
    }
}
\tcbset{
    green/.style={
        colback=green!10!white,
        colframe=black,
        boxrule=1pt,
        arc=0mm,
        auto outer arc,
        left=1mm,
        right=1mm,
        top=1mm,
        bottom=1mm
    }
}

\newtheorem{example}{例题}[section] 

% 调整标题的垂直间距
\setlength{\droptitle}{-60pt}
\pretitle{\begin{center}\LARGE\bfseries\vskip -20pt}
\posttitle{\par\vskip -10pt\rule{\textwidth}{0.4pt}\vskip -10pt\end{center}}
\predate{\begin{center}\large\vskip -10pt}
\postdate{\par\vskip -10pt\rule{\textwidth}{0.4pt}\vskip -20pt\end{center}}

\title{信号分析与处理}
\author{zjm54321}
\date{\today}

\begin{document}
    \maketitle

    \section{单位信号}
    
    \begin{tcolorbox}[green]
    信号之间的关系:
        \begin{equation*}
            \begin{aligned}
                r^\prime (t) &= u(t)\\
                u^\prime (t) &= \delta (t)\\
                \delta^\prime (t) &= \delta^\prime (t)\\
            \end{aligned}
        \end{equation*}
    \end{tcolorbox}

    \subsection{单位冲激信号的性质}

    \subsubsection{冲激信号}

    筛选特性:
    $$ f(t)\delta (t-t_0) = f(t_0)\delta (t-t_0) $$

    抽样特性:
    $$ \int_{-\infty}^{\infty}f(t)\delta (t-t_0)dt = f(t_0) $$

    \subsubsection{冲激偶信号}

    定义:
    $$ \delta^\prime (t) = \frac{d \delta (t)}{d t} $$

    筛选特性:
    $$ f(t)\delta^\prime (t-t_0) = f(t_0)\delta^\prime (t-t_0) -f^\prime (t_0)\delta(t-t_0) $$

    抽样特性:
    $$ \int_{-\infty}^{\infty}f(t)\delta^\prime (t-t_0)dt = -f^\prime (t_0) $$


    \begin{tcolorbox}[blue]
        \begin{example}
            求下列表达式的函数值:
            $$\int_{-\infty}^{\infty} e^{-j\omega t} [\delta (t) - \delta (t-t_0)] dt$$

            解:

            \begin{equation*}
                \begin{aligned}
                     \int_{-\infty}^{\infty} e^{-j\omega t} [\delta (t) - \delta (t-t_0)] dt\\
                    &= \int_{-\infty}^{\infty} e^{-j\omega t}\delta (t-0) dt - \int_{-\infty}^{\infty} e^{-j\omega t}\delta (t-t_0) dt\\
                    &= e^{-j\omega 0} - e^{-j\omega t_0}\\
                    &= 1 - e^{-j\omega t_0}\\
                \end{aligned}
            \end{equation*}
        \end{example}
    \end{tcolorbox}

    \subsection{单位阶跃信号的性质}

    \subsection{指数信号的积分与卷积}

    \section{信号变化}

    \subsection{信号的时间轴展缩、翻转、平移操作}
    步骤:
    \begin{enumerate}
        \item 展缩
        \item 翻转
        \item 平移
    \end{enumerate}

    \begin{tcolorbox}[blue]
        \begin{example}
            已知连续时间信号$x(t)$,画出$x(2-\frac{t}{3})$的波形

            解:
            \begin{enumerate}
                \item 展缩:$x(\frac{t}{3})$
                \item 翻转:$x(-\frac{t}{3})$
                \item 平移:$x(2-\frac{t}{3})$
            \end{enumerate}
        \end{example}
    \end{tcolorbox}

    \subsection{微分方程求解}
    求系统的完全响应:
    \begin{enumerate}
        \item 由已知方程写出齐次方程
        \item 由齐次方程写出特征方程
        \item 解出特征根$s_1,s_2,\cdots$
        \item 根据特征根写出齐次解:$y_h(t)$
        \begin{enumerate}
            \item 特征根是不等实根时, $y_h(t) = k_1e^{s_1t}+k_2e^{s_2t}+\cdots$
            \item 特征根是相等实根时, $y_h(t) = k_1e^{st}+k_2e^{st}+k_2t^2e^{st}+\cdots$
            \item 特征根是共轭虚根时, $y_h(t) = e^{s_1t}(k_1\cos(\omega t) + k_2\sin(\omega t))+\cdots$
        \end{enumerate}
        \item 由输入$f(t)$的形式,设出方程的特解:$y_p(t) = cf(t)$
        \item 将特解带入原方程求出$c$ 
        \item 求解全解$y(t) = y_h(t)+y_p(t)$
        \item 解初始化条件代入可得系数
        \item 写出$y(t)$
    \end{enumerate}

    \begin{tcolorbox}[blue]
        \begin{example}
            求解所$y^{\prime\prime}(t)+4y^\prime(t)+3y(t)=x^\prime(t)+2x(t)$描述的系统的单位冲激响应。

            解:
            \begin{enumerate}
                \item 系统的齐次方程为 $y^{\prime\prime}(t)+4y^\prime(t)+3y(t)= 0$
                \item 系统的特征方程为 $\lambda^2 + 4\lambda + 3 = 0 $
                \item 特征根为 $\lambda_1 = -1 , \lambda_2 = -3$
                \item 齐次解为 $y_h(t) = k_1e^{-t} + k_2e^{-3t}$
                \item 由单位冲激响应,故$y(t) = y_h(t)u(t) , x(t) = \delta(t)$
                \item 特解为 $y_p(t)=\delta^\prime(t)+2\delta(t)$
                \item 代入原方程,求得$k_1=\frac{1}{2} , k_2 = \frac{1}{2}$
                \item $y_h(t) = (\frac{1}{2}e^{-t}+\frac{1}{2}e^{-3t})u(t)$

            \end{enumerate}
        \end{example}
    \end{tcolorbox}

    \section{傅里叶变化}

    \subsection{傅里叶级数系数的计算}  
    角频率$\omega= \frac{2\pi}{T}$
    \begin{equation*}
        F_n=\frac{1}{T}\int_{-\frac{T}{2}}^{\frac{T}{2}}f(t)e^{-jn\omega t}dt
    \end{equation*}

    \begin{tcolorbox}[green]
        正弦余弦欧拉公式展开:
        \begin{equation*}
            \begin{aligned}
                \sin \omega t &= \frac {e^{j \omega t}- e^{-j \omega t}}{2 j} \\
                \cos \omega t &= \frac {e^{j \omega t}+ e^{-j \omega t}}{2  } \\
            \end{aligned}
        \end{equation*}
    \end{tcolorbox}

    \subsection{周期信号的频谱分析}

    \section{单边指数信号的傅里叶变换}

    傅里叶变换的性质:
    \begin{enumerate}
        \item 展缩:     $f(at) \rightarrow  \frac{1}{|a|} F(j \frac{\omega}{a})$
        \item 时移:     $f(t-t_0) \rightarrow F(j\omega) e^{-j\omega t_0}$
        \item 频移:     $f(t)e^{j\omega t} \rightarrow F(j(\omega-\omega_0))$
        \item 卷积:     $f_1(t) \ast f_2(t) \rightarrow F_1(j\omega) F_2(j\omega) $
        \item 时域微分: $\frac{d^n f(t)}{d t^n} \rightarrow (j\omega)^n F(j\omega)$
        \item 时域积分: $\int_{-\infty}^{t}f(t)dt \rightarrow \frac{1}{j\omega}F(j\omega)+\pi F(0) \delta(\omega)$
    \end{enumerate}

    常用傅里叶变换:
    \begin{enumerate}
        \item $f(t)              \rightarrow F(j\omega)$
        \item $\delta(t)         \rightarrow 1$
        \item $1                 \rightarrow 2\pi \delta(\omega)$
        \item $e^{j\omega t}     \rightarrow 2\pi \delta(\omega-\omega_0)$
        \item $e^{-\alpha t}u(t) \rightarrow \frac{1}{\alpha + j\omega}$
    \end{enumerate}

    \begin{tcolorbox}[blue]
        \begin{example}
            求$x(t) = e^{-3(t-1)}\delta^\prime(t-1)$的傅里叶变换。

            解:
            \begin{equation*}
                \begin{aligned}
                    x(t)
                    &= e^{-3(t_0-1)}\delta^\prime(t-1) - [-3e^{-3(t_0-1)}]\delta(t-1)\\
                    &= \delta^\prime(t-1) + 3\delta(t-1)\\
                    X(j\omega)
                    &= [(j\omega) \cdot 1 + 3 \cdot 1] \cdot e^{-j\omega t_0}\\
                    &= (j\omega + 3)e^{-j\omega}\\
                \end{aligned}
            \end{equation*}
        \end{example}
    \end{tcolorbox}

    \begin{tcolorbox}[blue]
        \begin{example}
            求$x(t) = e^{-2t}u(t+1)$的傅里叶变换。

            解:
            \begin{equation*}
                \begin{aligned}
                    x(t)
                    &= e^2 \cdot e^{-2(t+1)} u(t+1)\\
                    X(j\omega)
                    &= e^2 \cdot \frac{1}{-2+j\omega} e^{-j\omega(-1)}\\
                    &= \frac{e^{2+j\omega}}{-2+j\omega}\\
                \end{aligned}
            \end{equation*}
        \end{example}
    \end{tcolorbox}

    \section{左边等的指数序列的DTFT}
    \begin{equation*}
        X(e^{j\omega}) = \sum_{n= -\infty}^{\infty}x(n)e^{-j\omega n}
    \end{equation*}

    \begin{tcolorbox}[blue]
        \begin{example}
            求$x_1(n) = (\frac{1}{2})^n u(n+3)$的DTFT

            解:
            \begin{equation*}
                \begin{aligned}
                    X(e^{j\omega}) 
                    &= \sum_{n = - \infty}^{\infty}(\frac{1}{2})^n u(n+3) e^{-j\omega n}\\
                    &= \sum_{n = - 3}^{\infty}(\frac{1}{2})^n e^{-j\omega n}\\
                \end{aligned}
            \end{equation*}
        \end{example}
    \end{tcolorbox}

    \section{电路}

    \subsection{电路的系统函数求解}

    \subsection{无失真传输的频域条件}



\end{document}